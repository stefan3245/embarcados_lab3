\chapter{Arquitetura física}

\section{Uso de memória}

A Tabela \ref{tab:uso_memoria} apresenta o uso estimado das memórias RAM e flash do Kit LPC1768 para o desenvolvimento do sistema. Considerou-se na estimativa apenas os componentes que já estavam disponíveis ao projeto anteriormente (CMSIS RTOS e driver da UART). Pela análise, como a maior parte das memórias permanece livre, considera-se que a quantidade é suficiente para o desenvolvimento dos componentes restantes do sistema.

\begin{table}[h!]
	\caption{Uso estimado das memórias do Kit LPC1768.}
	\centering
	\begin{tabular}{|c|c|c|}
		\hline
		\textbf{} &\textbf{RAM (64 kB máx.)} & \textbf{Flash (512 kB máx.)} \\ \hline \hline
		CMSIS RTOS & 2600 B & 5400 B\\
		Driver da UART & 150 B & 800 B\\
		\hline
		\bf{Total ocupado} & 2750 B (2.68 kB) & 6200 B (6.05 kB)\\
		\bf{Total Livre} & 61.32 kB (95.81\%) & 505.95 kB (98.82\%) \\
		\hline
	\end{tabular}
	\label{tab:uso_memoria}
\end{table}


\section{Diagrama de objetos}

O diagrama apresentado na figura \ref{fig:objetos} representa os objetos e a comunicação entre eles no sistema.

\begin{figure}[h]
    \centering
    \includegraphics[width=0.8\columnwidth]{./figures/Objetos.png}
    \caption{Diagrama de objetos do sistema.}
    \label{fig:objetos}
\end{figure}

Os objetos Botões, Luzes, Portas, Sensores e Motor são objetos externos ao sistema, colocados no diagrama para indicar o interfaceamento externo do sistema.

Os sinais ``Botão pressionado'' indicam que o botão correspondente foi pressionado.

Os sinais ``Requisição atendida'' indicam que a requisição correspondente foi atendida e o sistema deve apagar a luz correspondente.

O sinal ``Porta aberta'' indica que as portas foram completamente abertas.

O sinal ``Porta fechada'' indica que as portas foram completamente fechadas.

Os sinais ``Chegou num andar'' indicam que o elevador chegou no andar correspondente.

Os sinais ``Comando para o elevador'' indicam um dos comandos possíveis para o elevador: Subir; Descer; Parar; Abrir portas; Fechar portas.

Os sinais ``Parou num andar'' indicam que o elevador parou no andar correspondente.

A tarefa Comunicação é responsável por fazer o interfaceamento interno ao sistema, através de um periférico de comunicação serial. Esta tarefa faz a leitura dos dados recebidos pela UART e transforma em sinais para serem usados pelas outras duas tarefas. Ela também recebe comandos para controle do elevador da tarefa ControleElevador e envia pela UART para o sistema externo.

A tarefa Enfileirador é responsável por colocar e remover as requisições em uma fila.

A tarefa ControleElevador é responsável por decidir sobre o movimento do elevador e a abertura e fechamento das portas.

\section{Projeto dos componentes}

A Figura \ref{fig:maq_estados} apresenta um diagrama de estados, contendo a visão geral do funcionamento do sistema. O estado ``Inicialização'' compreende todas as funções que são necessárias para iniciar a operação do sistema (como por exemplo, inicializar o driver da UART, sistema operacional, variáveis internas, entre outros). O estado ``Ocioso'' representa a situação em que o elevador não possui requisições pendentes, sendo que neste caso ele permanece parado no mesmo andar e com as portas abertas. 

Os estados ``Subindo'' e ``Descendo'' são muito similares internamente, ambos representado situações em que o elevador está atendendo a requisições de usuários. A diferença está essencialmente na direção de movimento do elevador. Os seus sub-estados especificam o modo de operação do elevador para atender às demandas de deslocamento até os andares requisitados. Todas as requisições de usuários são enfileiradas por prioridade, em uma tarefa separada (enfileirador). O próximo andar que o elevador deve ir fica sempre no topo da fila, sendo que as a maioria das decisões e transições nos estados e sub-estados depende de qual é este andar.

A comunicação com o computador (executanto a simulação de elevador) está presente de forma implícita no diagrama, o que inclui o recebimento de requisições de botões, recebimento de informações de posição do elevador e envio de comandos ao elevador e portas.

\begin{figure}[h]
    \centering
    \includegraphics[width=1\columnwidth]{./figures/maq_estados.png}
    \caption{Diagrama de estados geral do sistema.}
    \label{fig:maq_estados}
\end{figure}
