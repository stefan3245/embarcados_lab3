
% Proposta TCC1 - Professora Ana Cristina B. Kochem Vendramin (cristina@dainf.ct.utfpr.edu.br, criskochem@utfpr.edu.br)

\documentclass{normas-utf-tex_07_2012} %Estilo de Formato criado pelo Prof. Hugo Vieira Neto (hvieir@utfpr.edu.br)
%Se você ainda não conhece o latex, comece olhando o site do Prof. Hugo --> http://pessoal.utfpr.edu.br/hvieir/orient/

%\documentclass[twoside,openright]{normas-utf-tex} %openright = o capitulo comeca sempre em paginas impares

\usepackage[alf,abnt-emphasize=bf,bibjustif,recuo=0cm, abnt-etal-cite=2, abnt-etal-list=99]{abntcite} %configuracao das referencias bibliograficas.
\usepackage[brazil]{babel} % pacote portugues brasileiro
\usepackage[utf8]{inputenc} % pacote para acentuacao direta
\usepackage{amsmath,amsfonts,amssymb} % pacote matematico
\usepackage[pdftex]{graphicx} % pacote grafico
\usepackage{times} % fonte times
\usepackage{a4wide}
\usepackage[a4paper]{geometry} %define papel a4...
\geometry{left=3cm,right=2cm,top=3cm,bottom=2cm} % ...e suas margens
\usepackage{tabularx,multirow,longtable} %pacotes para mesclar linhas/colunas; tabelas grandes
\usepackage{fancyhdr} % altera cabeçalhos
\usepackage[T1]{fontenc} % acentuação direto no texto
\usepackage{ae} % “Almost European”: aumenta a qualidade do pdf final
\usepackage{rotating} % faz rotações de tabelas e figuras
\usepackage{indentfirst} % tabula a primeira linha do parágrafo
\usepackage[hang,small,bf]{caption} % legendas; nome "tabela" ou "figura" em negrito
\usepackage{caption3}
\usepackage{setspace} % ajuste do espaço entre-linhas
\usepackage{algorithm}
\usepackage{subfig}
\usepackage{float} % posicionamento das figuras
\usepackage{subfloat}
\usepackage{arydshln}
\usepackage{enumitem}
\makeatletter
\renewcommand{\ALG@name}{Algoritmo}
\usepackage{algorithmic}

%Italico e negrito customizados (\it e \bf)
\renewcommand{\it}[1]{\textit{#1}}
\renewcommand{\bf}[1]{\textbf{#1}}

%para a hifenacao funcionar é necessario fazer a seguinte modificacao
%Clique em Iniciar -> Programas -> MikTeX -> MiKTeX options Vá em Languages e marque a caixa português.

%hifenacoes customizadas
\hyphenation{mo-de-lar re-no-vá-veis re-pre-sen-tar re-pre-sen-ta-ção la-te-rais res-pec-ti-vos re-a-ção re-a-li-zan-do o-pe-ra-ções cons-tan-te di-fe-ren-tes tem-pe-ra-tu-ra ex-tre-mi-da-de ter-mo-di-nâ-mi-ca trans-es-te-ri-fi-ca-ção In-ter-net}

%%%% Comandos introduzidos para controlar as alteracoes no fonte%%%%%%%%%%%%%%%%%
% assigning colors to comments of each author
\usepackage{color}
\newcommand{\cris}[1]{\textcolor{blue}{#1}} %texto final
\newcommand{\crisC}[1]{[\textcolor{blue}{#1}]} % comentário entre colchetes
%%%%%%%%%%%%%%%%%%%%%%%%%%%%%%%%%%%%%%%%%%%%%%%%%%%%%%%%%%%%%%%%%%%%%%%%%%%%%%%%%%%%%%

% ---------- Preambulo ----------
\instituicao{Universidade Tecnol\'ogica Federal do Paran\'a} % nome da instituicao
\programa{Departamento Acad\^emico de Inform\'atica} % nome do programa ou departamento
\area{Curso de Engenharia de Computa\c{c}\~ao} % área ou curso

\documento{APS} % [Trabalho de Conclus\~ao de Curso] ou [Disserta\c{c}\~ao] ou [Tese]
\nivel{Gradua\c{c}\~ao} % [Gradua\c{c}\~ao], [Especializa\c{c}\~ao], [Mestrado] ou [Doutorado]
\titulacao{Engenheiro} % [Engenheiro], [Tecn\'ologo], [Bacharel], [Mestre] ou [Doutor]

\titulo{\MakeUppercase{Documento de especificação e estudo da plataforma}} % titulo do trabalho em portugues
\title{\MakeUppercase{}} % titulo do trabalho em ingles

\autor{Marcelo Teider Lopes} % autor do trabalho
\autordois{Stefan Campana Fuchs}
% \cita{SOBRENOME, Nome Autor} % sobrenome (maiusculas), nome do autor do trabalho

\palavraschave{Incluir as palavras-chave em português} %substituir pelas palavras-chave relacionadas ao seu tema de pesquisa
\keywords{Incluir as palavras-chave em inglês} % incluir palavras-chave em inglês

\comentario{\UTFPRdocumentodata\ de Sistemas Embarcados apresentada ao \UTFPRprogramadata\ da \ABNTinstituicaodata\ como requisito parcial para obten\c{c}\~ao do título de ``Engenheiro em Computa\c{c}\~ao''.} %\\ \'Area de Concentra\c{c}\~ao: \UTFPRareadata.}

\orientador{Prof. Douglas Paulo Bertrand Renaux} % <- no caso de orientadora, usar esta sintaxe
% \coorientador[Co-orientadora:~]{~} % <- no caso de co-orientadora, usar esta sintaxe

\local{Curitiba} % cidade
\data{2014} % ano

\setcounter{tocdepth}{4} % para numeração das seções
\setcounter{secnumdepth}{4}

\numberwithin{equation}{chapter} %
%\numberwithin{table}{chapter} %
%\numberwithin{figure}{chapter} %

\makeatletter  %% this is crucial
 \renewcommand\subsubsection{\@startsection{subsubsection}{3}{\z@}%
                        {-2\p@ \@plus -0.5\p@ \@minus -0.5\p@}%
                        %{8\p@ \@plus 4\p@ \@minus 4\p@}%     <-- this is copied from the subsection command
                        {2\p@ \@plus 1\p@ \@minus 1\p@}%     <-- this is copied from the subsection command
                        {\normalfont\normalsize\bfseries\boldmath
                         \rightskip=\z@ \@plus 8em
 \pretolerance=10000 }}
\makeatother   %% this is crucial

%---------- Inicio do Documento ----------
\begin{document}

\capa % geracao automatica da capa

\folhaderosto % geracao automatica da folha de rosto

% dedicatória (opcional)
%\begin{dedicatoria}
% Texto da dedicat\'oria.
%\end{dedicatoria}

% agradecimentos (opcional)
%\begin{agradecimentos}
% Texto dos agradecimentos.
%\end{agradecimentos}

% epigrafe (opcional)
%\begin{epigrafe}
%\end{epigrafe}

% %resumo
% \begin{resumo}
% 
% 
% Incluir o resumo aqui
% 
% 
% \end{resumo}
% 
% %abstract
% \begin{abstract}
% 
% 
% Include the abstract here
% 
% 
% \end{abstract}

% listas que recomenda-se a partir de 5 elementos
\listadefiguras % geracao automatica da lista de figuras
%Elaborado de acordo com a ordem apresentada no texto, com cada item designado por seu nome específico, acompanhado do respectivo número da página.

%\listadetabelas % geracao automatica da lista de tabelas
%Elaborado de acordo com a ordem apresentada no texto, com cada item designado por seu nome específico, acompanhado do respectivo número da página.

%\listadesiglas % geracao automatica da lista de siglas
%Constituída de uma relação alfabética das abreviaturas e siglas utilizadas no texto, seguido das palavras ou expressões correspondentes grafadas por extenso. Utilizada apenas se houver siglas.

%\listadesimbolos % geracao automatica da lista de simbolos
%Elaborado de acordo com a ordem apresentada no texto, seguido do significado correspondente. Utilizada apenas se houver símbolos.

% sumario
\sumario % geracao automatica do sumario

%---------- Inicio do Texto ----------

\setcounter{page}{4} % *** Necessário arrumar manualmente antes de imprimir

% Recomendo a criação de um arquivo .tex para cada capítulo do seu trabalho. Isso facilitará, principalmente, quando vários usuários estiverem editando o mesmo documento.
% Neste momento, colocarei tudo aqui no mesmo .tex.
% Para usar arquivos diferentes, basta descomentar os comandos abaixo.
%\input{Introducao.tex}
%\input{LevantamentoBibiografico.tex}
%\input{Metodologia.tex}
%\input{Recursos.tex}
%\input{Viabilidade.tex}
%\input{Contexto.tex}
%\input{Conclusao.tex}

% Criando o capítulo Introdução. Caso crie um arquivo .tex, basta inserir o texto abaixo dentro do arquivo.
% O label serve para referenciar o capítulo quando necessário
\chapter{Especificação do produto} \label{cap:especificacao_produto}

\section{Especificação funcional}

Os requisitos funcionais levantados para o \it{software} são:

\begin{enumerate}[label=RF \arabic* -- , ref=\arabic*]
	\item O sistema deverá ligar a luz de um botão quando este for pressionado.
  \item O sistema deverá desligar a luz de um botão quando o elevador parar no andar correspondente ao botão.
  \item O sistema deverá abrir as portas quando o elevador parar em um andar.
  \item O sistema deverá impedir que as portas sejam abertas quando o elevador não estiver posicionado em um andar.
  \item O sistema deverá impedir que as portas sejam abertas quando o elevador estiver em movimento.
  \item O sistema deverá impedir que o elevador se movimente quando as portas estiverem abertas.
  \item O sistema deverá atender a requisições de mudança de andar, feitas através dos botões.
  \begin{enumerate}[label*=\arabic*]
    \item O sistema deverá enfileirar as requisições de mudança de andar.
    \item O sistema deverá manter apenas uma requisição de cada andar na fila.
    \item O sistema deverá dar prioridade à requisições feitas com os botões internos.
    \item O sistema deverá dar prioridade às requisições dos andares mais altos feitas com botões externos de descida.
    \item O sistema deverá dar prioridade às requisições dos andares mais baixos feitas com botões externos de subida.
    \item O sistema deverá parar o elevador quando ele estiver passando por um andar que o botão interno correspondente tenha sido pressionado.
    \item O sistema deverá parar o elevador quando ele estiver passando por um andar que o botão externo tenha sido pressionado, se estiver indo na mesma direção que a requisição foi feita.
    \end{enumerate}
  \item O sistema deverá esperar no mínimo 5 segundos após as portas terem sido fechadas antes de fechar elas novamente.
  \item O sistema deverá esperar meio segundo após fechar as portas antes de deslocar o elevador.
  \item O sistema deverá esperar meio segundo após o elevador parar antes de abrir as portas.
  \item O sistema deverá esperar no mínimo 2 segundos para fechar a porta após um botão interno ser pressionado.
\end{enumerate}

Os requisitos não-funcionais levantados para o \it{software} são:

\begin{enumerate}[label=RNF \arabic* -- , ref=\arabic*]
	\item A
	\item B
\end{enumerate}



\chapter{Estudo da plataforma} \label{cap:estudo_plataforma}

\section{UART}

A UART é o periférico presente no Kit LPC1768 responsável por efetuar a comunicação serial. Ela efetua a conversão de dados de I/O entre a interface serial e a CPU, gerenciando o processo de transmissão e recepção de dados. A UART possui filas de recepção e transmissão, configuráveis, permitindo a geração de interrupções por número de caracteres na fila ou por \it{timeout}. 

Os registradores utilizados para a inicialização da UART são utilizados como se segue (considerando-se que a UART0 da placa foi utilizada no projeto). O registrador PCONP controla a alimentação da UART, sendo que o bit 3 deve ser ativado para ligá-la. O registrador PCLKSEL0 controla a saída do clock para a UART, sendo que os bits 6:7 devem ser habilitados para que o clock seja ativado.








% A UART executa conversões de dados serial e paralelo entre um periférico e a CPU, permite também guardar em uma FIFO de dados de recepção e transmissão, além de possuir capacidade de controle e interrupção do sistema, o que diminui o overhead em software para gerenciar a comunicação. A base da troca de dados é um clock, que é configurado conforme medidas de baudrate específicas, e que especifica o momento em que os sinais recebidos pela serial são válidos. Um bloco de comunicação serial é caracterizado, em geral, por um bit de início, 5 a 8 bits de dados, 1 bit de paridade opcional, terminado por 1 a 2 bits de parada [8].

% No LandTiger há suporte para 2 interfaces RS-232, nomeadas COM1 e COM2, associados, respectivamente, as UART0 e UART2 do LPC1768.Para que sejam utilizados como comunicação UART, os jumpers JP6 e JP7 devem estar desativados, pois as portas podem ser utilizadas para ISP. A COM1 está conectada ao UART0_TX (P0.2) e UART0_TX (P0.3), enquanto a COM2 está conectada a UART2_TX (P0.10) e UART_RX (P0.11).

% Com relação ao software, um driver UART foi desenvolvido pela equipe anteriormente numa plataforma LPC 1343 e portada para o LPC 1768.
% A alimentação da UART é realizada através do registrador PCONP (Power Control for Peripherals Register), ativando o bit 3 (PCUART0). O clock para o periférico é habilitado ativando os bits 7:6 (PCLK_UART0) do PeripheralClockSelectionRegister0 (registrador PCLKSEL0). Para configurar o baudrate é utilizado o registrador U0LCR(UART0 Line Control Register), com o bit DLAB (Divisor Latch Access Bit) definido em 1 (é o bit 7 do U0LCR) . Essa configuração habilita os registradores DLL (Divisor Latch LSB register) e DLM (Divisor Latch MSB register) que determinam o baudrate da UART e o baudrate fracionário no registrador divisor fracionário (Fractional Divider Register - FDR). Durante a configuração do U0LCR configura-se o tamanho da Word (bits 1:0), seleciona-se a quantidade de stop bits (bit 2), habilita-se ou desabilita-se a paridade (bit 3), selecionando também seu tipo (bits 5:4). Em seguida, habita-se a FIFO da UART no bit 0 do registrador U0FCR. Os pinos da UART são selecionados do registrador PINSEL e seus modos nos registradores PINMODE. As interrupções da UART são habilitadas no registrador U0LCRcom o bit DLAB0.


%---------- Referencias ----------

% \chapter{Referências Bibliográficas}

% \nocite{sumo_overview} %remover depois, está aqui só para conseguir compilar (precisa ter pelo menos 1 referencia)!
% \bibliography{Referencias}

%---------- Apêndices e Anexos ----------
% % Apêndice
% \appendix
% %\input{Apendice.tex} % Para criar um arquivo separado para o Apêndice, descomente essa linha e comente a linha abaixo
% \chapter{Título do Apêndice}
% 
% % Anexo
% \appendix
% \renewcommand{\appendixname}{Anexo} % Para criar Anexo ao invés de Apêndice
% %\input{Anexo.tex} % Para criar um arquivo separado para o Anexo, descomente essa linha e comente a linha abaixo
% \chapter{Título do Anexo}

\end{document}
