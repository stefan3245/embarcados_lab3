
\section{UART}

A UART é o periférico presente no Kit LPC1768 responsável por efetuar a comunicação serial. Ela efetua a conversão de dados de I/O entre a interface serial e a CPU, gerenciando o processo de transmissão e recepção de dados. A UART possui filas FIFO de recepção e transmissão, configuráveis, permitindo a geração de interrupções por número de caracteres na fila ou por \it{timeout}. 

Os registradores utilizados para a inicialização da UART são utilizados como se segue (considerando-se que a UART0 da placa foi utilizada no projeto): O registrador PINSEL é utilizado para selecionar os pinos da UART, e o PINMODE é utilizado para configurar os modos dos pinos. O registrador PCONP controla a alimentação da UART, sendo que o bit 3 deve ser ativado para ligá-la. O registrador PCLKSEL0 controla a saída do clock para a UART, sendo que os bits 7:6 devem ser habilitados para que o clock seja ativado. O baudrate é ativado pelo registrador U0LCR, sendo que o bit 7 deve ser setado. Os registradores DLL (Divisor Latch LSB register) e DLM (Divisor Latch MSB register) podem então ser utilizados para configurar o valor de baudrate, de acordo com a fórmula definida na documentação do Kit LPC1768. 

O registrador U0LCR é então utilizado para configurar o tamanho da word (bits 1:0), stop bits (bit 2) e paridade (bit 3 para habilitar ou não, bits 5:4 para selecionar o tipo). O registrador U0FCR é utilizado para habilitar e limpar as FIFOs de RX e TX (setando os bits 2:0) e para configurar o número de caracteres para trigger de interrupção (bit 7). As interrupções da UART são então ativadas setando-se os bits 1:0 de U0LCR.
